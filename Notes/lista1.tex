\section{Ejercicios}
\subsection{Lista N$^{\circ}1$}
\begin{enumerate}[font={\bfseries},label={\arabic*.}]
\item Un número racional $a/b$ con $(a,b)=1$ se llama \emph{fracción reducida}. Si la suma de dos fracciones reducidas es un entero, es decir, si $(a/b)+(c/d)=n$. Demostrar que entonces $|b|=|d|$.

\item Si $(a,b)=1$, entonces $(a+b,a-b)$ o es 1 o es 2.

\item Si $(a,b)=1$, entonces $(a+b,a^{2}-ab+b^{2})$ o es 1 o es 3.

\item Si $(a,b)=1$, entonces $(a^{n},b^{k})=1$ para todo $n\geq1, k\geq1$.

\item Un entero se llama \emph{sin cuadrados} si no es divisible por el cuadrado de ningún primo. Probar que, para cada $n\geq1$, existen $a>0$ y $b>0$, unívocamente determinados, tales que $n=a^{2}b$, en donde $b$ es sin cuadrados.

\item Probar que $\dfrac{21n+4}{14n+3}$ es irreducible para todo número natural $n$.

\item Sean $\{a,b,x,y\}\subset\mathbb{N}$. Si $(a,b)=1$ y $ab=c^{n}$, probar que $a=x^{n}$ y $b=y^{n}$ para algunos $x,y$ enteros positivos.

\item Hallar $\left(a^{2^{m}}+1,a^{2^{n}}+1\right)$ en función de $a$.

\item Sean $\{a,b,x,y\}\subset\mathbb{N}$. Si $(a,b)=1$ y $x^{a}=y^{b}$ entonces probar que $x=n^{b}$ e $y=n^{a}$ para algún entero positivo.

\item Si $\{a,m,n\}\subset\mathbb{N}$ con $a>1$, probar que $\left(a^{m}-1,a^{n}-1\right)=a^{(m,n)}-1$.

\item Sea $n$ un entero positivo y sea $S$ un conjunto de enteros positivos menores o iguales a $2n$ tal que si $a$ y $b$ están en $S$ y $a$ y $b$ son diferentes, entonces $a$ no divide a $b$. Hallar el máximo número de elementos de $S$.

\item Hallar todos los pares de enteros positivos $(a,b)$ tales que $a\divides b+1$ y $b\divides a+1$.

\item Hallar todos los pares de enteros positivos $(a,b)$ tales que $a\divides8b+1$ y $b\divides8a+1$.

\item Halle todos los números enteros positivos $n$ tales que el conjunto $\{n,n+1,n+2,n+3,n+4,n+5\}$ puede ser particionado en dos subconjuntos de modo que el producto de los números en cada subconjunto sea igual.

\item Sea $m$ y $n$ números enteros tales que:
\[\frac{m}{n}=1-\frac{1}{2}+\frac{1}{3}-\frac{1}{4}+\cdots-\frac{1}{1318}+\frac{1}{1319}\]

Probar que $m$ es divisible por 1979. Ayuda: 1979 es un número primo.
\end{enumerate}

\section{Solución de la lista N$^{\circ}$1}

\begin{enumerate}[font={\bfseries},label={\arabic*.}]
\item
	\begin{enumerate}[font={\bfseries}]
	\item Dado $n=\dfrac{ad+bc}{bd}$, con $b$ y $d$ que dividen a $ad+bc$. Esto significa que $b\divides bc$ y $d\divides bc$, pero $\mcd(a,b)=\mcd(c,d)=1$. se tiene que $b\divides b$ y $d\divides b$. Por lo tanto, $|b|=|d|$.
	
	\item Sea $k\in\mathbb{Z}$ y $k=\dfrac{a}{b}+\dfrac{c}{d}$, entonces $\dfrac{c}{d}=k-\dfrac{a}{b}=\dfrac{ck-a}{b}$.
	
	Supongamos por contradicción que $\mcd(kb-a,b)\neq1$. Además, si $\mcd(kb-a,b)=n$, entonces $n\divides (kb-a)$ y  $n\divides b$.
		
	\[kb-a=n\ell_1\quad\wedge\quad b=n\ell_2\]
	\[k(n\ell_2)-a=n\ell_1\]
	\[n(k\ell_2-\ell_1)=a\]
	Pero de la última ecuación, se infiere que $n\divides a$ y $n\divides b$, lo cual es una contradicción.
	\end{enumerate}
	
	\item Si el $\mcd(a,b)=1$ y el $\mcd(a+b,a-b)=d$, entonces:
	
	\begin{align*}
	d\divides a+b\;&\wedge\; d\divides a-b&\implies d&\divides (a+b)+(a-b)\;&\wedge\; d&\divides (a+b)-(a-b)&\\
	d\divides a+b\;&\wedge\; d\divides a-b&\implies d&\divides 2a\;&\wedge\; d&\divides 2b&
	\end{align*}
	Por lo tanto $d\divides \mcd(2a,2b)\implies d\divides 2\mcd(a,b)\implies d\divides 2$, así, $d=1\vee2$.
	
	\item Si el $\mcd(a,b)=1$ y el $\mcd(a+b,a^{2}-ab+b^{2})=d$, entonces:
	
	\begin{align*}
	d\divides a+b\;&\wedge\; d\divides a^{2}-ab+b^{2}&\implies d&\divides {(a+b)}^{2}\;&\\
	d\divides a+b\;&\wedge\; d\divides a^{2}-ab+b^{2}&\implies d&\divides a^{2}+2ab+b^{2}\;&\\
	d\divides a+b\;&\wedge\; d\divides a^{2}-ab+b^{2}&\implies d&\divides
	\cancel{a^{2}}+2ab+\cancel{b^{2}}-(\bcancel{a^{2}}-ab+\bcancel{b^{2}})\;&&\\
	d\divides a+b\;&\wedge\; d\divides a^{2}-ab+b^{2}&\implies d&\divides3ab
	&
	\end{align*}
	Por lo tanto, $d\divides 3a(a+b)\implies d\divides 3a^{2}+3ab$. De esto, $d\divides 3a^{2}$ y $d\divides 3ab$.
	
	\[d\divides\mcd(3a^{2},3ab)\implies d\divides|3a|\cdot\mcd(a,b)\implies d\divides 3a.\]
	
	Como $d\divides a+b\implies d\divides 3a+3b\implies d\divides 3b$. %pues d divide a 3a
	$d\divides\mcd(3a,3b)\implies d\divides 3\underbrace{\mcd(a,b)}_{\displaystyle1}\implies d\divides 3$. Así, $d=1\vee3$.
	
	\item Si el $\mcd(a,b)=1$ y el $\mcd(a^{n},b^{k})=d>1$, entonces $d\divides a^{n}$ y $d\divides b^{k}$. Sea $p$ un número primo de modo que $p\divides d$. Así:
	\begin{align*}
	p\divides a^{n}\;\wedge\;p\divides b^{k}\implies p\divides a\;\wedge\; p\divides b\implies p\divides\underbrace{\mcd(a,b)}_{\displaystyle1}
	\end{align*}
	¡Un número primo divide a 1! $(\implies\impliedby)$. $\therefore d=1$.
	
	\item Sea el conjunto $\mathbb{N}=\mathcal{A}\bigcup\mathcal{B}$, donde $\mathcal{A}\coloneq\{n\in\mathbb{N}\mid n\text{ es libre de cuadrados} \}$ y $\mathcal{B}\coloneq\mathcal{A}^{\complement}$.
	
	Para el caso 1: Sea $\gamma=1^{2}\cdot\gamma$, si $\gamma\in\mathcal{A}$.
	
	Sea $\theta\in\mathcal{B}$, entonces $\exists z^{2}\ni\theta=z^{2}\cdot\tau$.
	
	Supongamos que $\tau$ no es libre de cuadrados:
	\[\tau=n^{2}\cdot m \checkmark\]
	Al reemplazar resulta:
	\[\theta=z^{2}n^{2}m=(z\cdot n)^{2}\cdot m(\implies\impliedby)\]
	
	\item Por contradicción y combinación lineal
	\[\mcd(a,b)=1\;\text{y}\;a\cdot b=c^{n}\]
	donde $a$ y $b$ tienen las siguiente forma:
	\[a={p}^{\alpha_1}_{1}{p}^{\alpha_2}_{2}\cdots{p}^{\alpha_k}_{k}\quad\text{y}\quad b={q}^{\beta_1}_{1}{q}^{\beta_2}_{2}\cdots{q}^{\beta_\ell}_{\ell}.\]
	¡Recuerde que a y b no tiene factores comunes!
	Así, multiplicando $a$ y $b$:
	\[
	a\cdot b={p}^{\alpha_1}_{1}{p}^{\alpha_2}_{2}\cdots{p}^{\alpha_k}_{k}\cdot{q}^{\beta_1}_{1}{q}^{\beta_2}_{2}\cdots{q}^{\beta_\ell}_{\ell}=c^{n},
	\]
	donde
	\[c={p}^{\theta_1}_{1}{p}^{\theta_2}_{2}\cdots{p}^{\theta_k}_{k}\cdot{q}^{\phi_1}_{1}{q}^{\phi_2}_{2}\cdots{q}^{\phi_\ell}_{\ell}\]
	\[c^{n}={p}^{\theta_1\cdot n}_{1}{p}^{\theta_2\cdot n}_{2}\cdots{p}^{\theta_k\cdot n}_{k}\cdot{q}^{\phi_1\cdot n}_{1}{q}^{\phi_2\cdot n}_{2}\cdots{q}^{\phi_\ell\cdot n}_{\ell}\]
	
	Por comparación obtenemos las siguientes igualdades:
	\[
	\alpha_1=\theta_1\cdot n,\;\alpha_2=\theta_2\cdot n,\;\cdots,\alpha_k=\theta_k\cdot n.%Tratar de reproducirlo con un foreach con Tikz.
	\]
	\[
	\beta_1=\phi_1\cdot n,\;\beta_2=\phi_2\cdot n,\;\cdots,\beta_\ell=\phi_\ell\cdot n.
	\]
	Así, $a={\left(\underbrace{{p}^{\theta_1}_{1}{p}^{\theta_2}_{2}\cdots{p}^{\theta_k}_{k}}_{\displaystyle x}\right)}^{n}$ y $b={\left(\underbrace{{q}^{\phi_1}_{1}{q}^{\phi_2}_{2}\cdots{q}^{\phi_\ell}_{\ell}}_{\displaystyle y}\right)}^{n}$.
	
	\item Sea $g=\mcd(\mathfrak{a}^{2^{m}}+1,\mathfrak{a}^{2^{n}}+1)$. Se define la aplicación $f_{\mathfrak{a}}$ con la siguiente regla de correspondencia como sigue:
	\[\begin{aligned}
	f_{\mathfrak{a}}\colon\mathbb{N}&\longrightarrow\mathbb{N}\\
	k&\longmapsto\mathfrak{a}^{2^{k}}+1
	\end{aligned}\]
	Así, con la notación apropiada, $g$ queda expresada como $\mcd\left(f_{\mathfrak{a}}(m),f_{\mathfrak{a}}(n)\right)$. Además $g\divides f_{\mathfrak{a}}(m)$ y $g\divides f_{\mathfrak{a}}(n)$.
	
	\begin{enumerate}
		\item Para el caso en que $m>n$:
		
		Ahora, calculemos:
		
		\begin{flalign*}
		f_{\mathfrak{a}}(m)-2
		&=\mathfrak{a}^{2^{m}}+1-2=\mathfrak{a}^{2^{m}}-1&\\
		&=\left(\mathfrak{a}^{2^{n}}\right)^{2^{m-n}}-1&\\
		&=\underbrace{\left(\mathfrak{a}^{2^{n}}+1\right)}_{\displaystyle f_{\mathfrak{a}}(n)}\left(\mathfrak{a}^{2^{(m-n-1)}}-1\right)&
		\end{flalign*}
		
		Así que $f_{\mathfrak{a}}(n)\divides f_{\mathfrak{a}}(m)-2$. Pero, como $g\divides f_{\mathfrak{a}}(n)\implies g\divides f_{\mathfrak{a}}(m)-2$.
		
		$\therefore g\divides -\cancel{f_{\mathfrak{a}}(m)}+2+\bcancel{f_{\mathfrak{a}}(m)}\implies g\divides 2$.
		
		\begin{enumerate}[label={$\bullet$}]
			\item Si $a$ es par, entonces $f_{\mathfrak{a}}(m)$ es impar y $g=1$.
			\item Si $a$ es impar, entonces $f_{\mathfrak{a}}(m)$ es par y $g=2$.
		\end{enumerate}
		
		\item Para el caso en que $m=n$:
		
		\item Para el caso en que $m<n$:
	\end{enumerate}
	
	\item Si el $\mcd(a,b)=1$ y los números $a$ y $b$ tiene la siguiente representación:
	\[a={p}^{\theta_1}_{1}{p}^{\theta_2}_{2}\cdots{p}^{\theta_k}_{k}=\prod_{i=1}^{k}{p_i}^{\alpha_i}\]
	\[b={q}^{\beta_1}_{1}{q}^{\beta_2}_{2}\cdots{q}^{\beta_\ell}_{\ell}=\prod_{i=1}^{\ell}{q_i}^{\beta_i}\]
	
	Pero $x^{a}=$
\end{enumerate}
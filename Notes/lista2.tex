\section{Ejercicios}
\subsection{Lista N$^{\circ}2$}
\begin{enumerate}[font={\bfseries},label={\arabic*.}]
\item Hallar todos los enteros positivos $n$ tal que:
\begin{multicols}{3}
\begin{enumerate}
	\item $\varphi(n)=n/2$.
	\item $\varphi(n)=\varphi(2n)$.
	\item $\varphi(n)=12$.
\end{enumerate}	
\end{multicols}

\item Probar que
\[\sum_{d^{2}\divides n}\mu(d)=\mu^{2}(n).\]
\item Probar que

\[\frac{n}{\varphi(n)}=\sum_{d\divides n}\frac{\mu^{2}(d)}{\varphi(d)}.\]

\noindent
Sea $d(n)$ el número de divisores positivos de $n$.

\item Probar que $d(n)$ es impar si, y solo si, $n$ es cuadrado perfecto.

\item Probar que

\[\prod_{t\divides n}t=n^{d(n)/2}.\]

\item Probar que

\[\sum_{t\divides n}{d(t)}^{3}={\left(\sum_{t\divides n}d(t)\right)}^{2}.\]

Sea $f$ una función aritmética multiplicativa.

\item Probar que:

\begin{enumerate}
	\item $f^{-1}(n)=\mu(n)f(n)$ para todo $n$ libre de cuadrados (o sea, producto de números primos distintos).
	\item $f^{-1}(p^{2})={f(p)}^{2}-f(p^{2})$.
\end{enumerate}

\item Probar que $f$ es completamente multiplicativa si y solo si $f^{-1}(p^{a})=0$ para todo primo $p$ y todos los enteros $a\geq2$.

\item Sea $P(n)$ el producto de los enteros positivos menores o iguales a $n$ y coprimos con $n$. Probar que

\[P(n)=n^{\varphi(n)}\prod_{d\divides n}{\left(\frac{d!}{d^{d}}\right)}^{\mu\left(n/d\right)}.\]
\end{enumerate}

\section{Solución de los ejercicios}


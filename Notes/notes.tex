\documentclass[oneside,a5paper]{memoir}

\usepackage[utf8x]{inputenc}
\usepackage[T1]{fontenc}
\usepackage{libertine}
\usepackage[spanish]{babel}
\spanishdatedel
\usepackage[usenames,dvipsnames,x11names,table,svgnames]{xcolor}
\usepackage[colorlinks=true,urlcolor=blue,linkcolor=black,anchorcolor=black,citecolor=black]{hyperref}
\usepackage{mathtools,amssymb,amsfonts,amsmath,amsthm,mathrsfs,bm,times,mathabx}
\usepackage[makeroom]{cancel}
\usepackage{multicol}
\usepackage{geometry}
\usepackage{enumitem}
%\geometry{
%	tmargin=1cm, 
%	bmargin=0.5cm, 
%	lmargin=0.5cm, 
%	rmargin=1cm,
%	headheight=1.5cm,
%	headsep=0.8cm,
%	footskip=0.5cm}
\hypersetup{pdfinfo={
		Title={Apuntes de clases de Teoría de números},
		Author={Jimmy Espinoza Palacios},
		Keywords={Divisibilidad, Congruencias, Funciones aritméticas},
		Subject={Teoría de números},
		Producer={TeXstudio 2.12.6},
		Creator={pdfTeX Version 3.14159265-2.6-1.40.18 TeX Live 2018 Debian}
}}
\newcommand*{\colorboxed}{}
\def\colorboxed#1#{%
	\colorboxedAux{#1}%
}
\newcommand*{\colorboxedAux}[3]{%
	% #1: optional argument for color model
	% #2: color specification
	% #3: formula
	\begingroup
	\colorlet{cb@saved}{.}%
	\color#1{#2}%
	\boxed{%
		\color{cb@saved}%
		#3%
	}%
	\endgroup
}
%\usepackage[citestyle=numeric,style=numeric,backend=biber]{biblatex}
%\addbibresource{num.bib}
\DeclareMathOperator{\mcd}{mcd}
\newcommand{\MVAt}{{\usefont{U}{mvs}{m}{n}\symbol{`@}}} % Se prefiere no usar el paquete marvosym cuando esté cargada mathabx.
\renewcommand{\qedsymbol}{$\blacksquare$}
\newtheorem{definition}{Definición}[chapter] %chapter, section
\newtheorem{theorem}{Teorema}[chapter] %chapter section
\newtheorem{remark}{Observación}[chapter]
\theoremstyle{definition}
\pagestyle{empty}

\author{\textcolor{DarkRed}{Estudiante Oromion}\thanks{E-mail: {\href{mailto:caznaranl@uni.pe}{caznaranl\MVAt uni.pe}}.}}

\title{\textcolor{DarkBlue}{Apuntes de Teoría de números}\thanks{Gracias al \href{https://web.facebook.com/GEMFCUNI/}{Grupo Estudiantil de Matemática} y al \href{http://imca.edu.pe/portal/index.php/es/}{Instituto de Matemática y Ciencias Afines}.}}
\date{\textcolor{DarkMagenta}{Actualizado a la fecha \today}}

\begin{document}
\maketitle
%Verano 2018
\chapter{Introducción}%Fecha: 11 de enero del 2018.

\begin{enumerate}[font={\bfseries},label={\roman*.}]

\item\label{pr:1} \emph{Principio de inducción matemática}
	
Sea $\mathcal{P}$ un conjunto de números naturales tal que

\begin{enumerate}
	\item $1\in\mathcal{P}$.
	\item Si $n\in\mathcal{P}\implies n+1\in\mathcal{P}$.
\end{enumerate}

$\therefore \boxed{\mathcal{P}=\mathbb{N}}$.

\item\label{pr:2} \emph{Principio del buen orden}

Si $\mathcal{A}$ es un conjunto no vacío de $\mathbb{N}$, entonces \emph{$\mathcal{A}$ posee un elemento mínimo}.

\end{enumerate}

\section{Divisibilidad}

\begin{definition}\label{def:1.1}

Sean $d$ y $n$ dos números enteros, se denotará

\[\boxed{d\text{ divide a }n\iff\text{ existe }c\in\mathbb{Z}\text{ tal que }n=c\cdot n}\]
como $a\divides n$.

\noindent
Si $d$ no divide a $n$, es decir, si $\forall c\in\mathbb{Z}\colon n\neq c\cdot d$, se denotará como $d\notdivides n$.

\end{definition}

\subsection{Propiedades de la operación $|$}

\begin{enumerate}[font={\bfseries},label={\arabic*)}]

\item\label{prop:1} $n\divides n$ para cualquier $n\in\mathbb{N}$ (Reflexividad).

\item\label{prop:2} Si $d\divides n$ y $n\divides m$, entonces $d\divides m$. (Transitividad).

\item\label{prop:3} Si $d\divides n$ y $d\divides m$, entonces $d\divides an+bm$ $\forall a,b\in\mathbb{Z}$.

\item\label{prop:4} Si $d\divides n$, entonces $ad\divides an$.

\item\label{prop:5} Si $ad\divides an$ con $a\neq0$, entonces $d\divides n$.

\item\label{prop:6} $1\divides n$ para cualquier $n\in\mathbb{N}$.

\item\label{prop:7} $n\divides 0$ para cualquier $n\in\mathbb{N}$.

\item\label{prop:8} Si $0\divides n$, entonces $n=0$.

\item\label{prop:9} Si $d\divides n$ y $n\neq0$, entonces $|d|\leq|n|$.

\item\label{prop:10} Si $d\divides n$ y $n\divides d$, entonces $|d|=|n|$.

\item\label{prop:11} Si $d\divides n$ con $d\neq0$, entonces $\left(\frac{n}{d}\right)\divides n$.

\end{enumerate}

\section{Máximo común divisor}

\begin{definition}\label{def:1.2}

Sean $a$, $b$ y $d$ números enteros. Si $d\divides a$ y $d\divides b$, entonces $d$ es un divisor común de $a$ y $b$.

\end{definition}

\begin{theorem}\label{teo:1.1}

Dados los números enteros $a$ y $b$, existe un divisor común $d$ de $a$ y $b$ de la forma $d=ax+by$ para cualesquiera $x,y\in\mathbb{Z}$.

\begin{proof}[Prueba:]
Por inducción matemática en $K=|a|+|b|$.
 
Si $K=0$, entonces $a=b=0$, esto es, $d=0\cdot a+0\cdot b$. \checkmark
	 
\noindent
Supongamos que se cumple para $K=0,1,\ldots,n-1$. (Hipótesis de inducción matemática).
	 
Demostraremos para $\boxed{K=n=|a|+|b|}$.
	 
\noindent
Sin pérdida de generalidad, suponga que $|a|\geq|b|$. Así, si $|b|=0$, entonces $b=0$ y $|a|=n\implies d=n=(1)(\pm1)+0\cdot b$.
	 
Si $|b|\geq1$, entonces para los números $|a|-|b|$ y $|b|$ se cumple la hipótesis:

\[\underbrace{|a|-|b|}_{\displaystyle\geq0}+|b|=|a|-\cancel{|b|}+\bcancel{|b|}=|a|<|a|+|b|=n.\]

Existe $d\in\mathbb{Z}$, $d\divides |a|-|b|$ y $d\divides |b|$. Además:

\begin{align*}
	d&=\left(|a|-|b|\right)x^{\prime}+|b|y^{\prime}&\forall x^{\prime},y^{\prime}\in\mathbb{Z}\\
	d&=|a|\underbrace{x^{\prime}}_{\displaystyle x^{\prime\prime}}+|b|\underbrace{y^{\prime}}_{\displaystyle y^{\prime\prime}}&\\
	d&=\underbrace{|a|}_{a,-a}x^{\prime\prime}+\underbrace{|b|}_{b,-b}y^{\prime\prime}&\\
	d&=a\underbrace{x^{\prime\prime}}_{\pm x^{\prime}}+b\underbrace{y^{\prime\prime}}_{\pm y^{\prime}}&
\end{align*}

\noindent
Pero $d\divides |a|$ y $d\divides |b|$, así $d\divides |a|-|b|$.

\noindent
$\therefore$ Esto cumple la condición.
\end{proof}

\end{theorem}

\begin{theorem}\label{teo:1.2}
Sean $a$ y $b$ números enteros, existe solo un número $d\in\mathbb{Z}$ tal que

\begin{enumerate}[font={\bfseries},label={\arabic*)}]

\item\label{teo1.2:1} $d\geq0$.

\item\label{teo1.2:2} $d\divides a$ y $d\divides b$.

\item\label{teo1.2:3} Si $e\divides a$ y $e\divides b$, entonces $e\divides d$ para cualquier $e\in\mathbb{Z}$.

\end{enumerate}

\begin{proof}[Prueba:]
Por la definición~\ref{def:1.2} y por el teorema~\ref{teo:1.1}, existe un $d$ con las siguientes propiedades:
\begin{multicols}{3}
$d\divides a$

$d\divides b$

$d=ax+by$
\end{multicols}

\noindent
Es claro que $-d$ también cumple esto. Elegimos $|d|=ax^{\prime}+by^{\prime}$ que cumpla \ref{teo1.2:1} y \ref{teo1.2:2}.

\noindent
Si $e\divides a$ y $e\divides b$, entonces de la propiedad~\ref{prop:3} $e\divides ax^{\prime}+by^{\prime}=|d|$.
	
\noindent
Así, $e\divides |d|$, en consecuencia $e\divides d$ y $|d|$ satisface~\ref*{teo1.2:3}.
	
\noindent
Si existiese un $d^{\prime}$ que cumpla \ref{teo1.2:1}, \ref{teo1.2:2} y \ref{teo1.2:3}, entonces de la afirmación \ref{teo1.2:3}:

\begin{equation}\label{eq:1.1}
d\divides a\text{ y }d\divides b\implies d\divides d^{\prime}.
\end{equation}
	
\noindent
De forma similar:
\begin{equation}\label{eq:1.2}
d^{\prime}\divides a\text{ y }d^{\prime}\divides b\implies d^{\prime}\divides d.
\end{equation}

\noindent
Pero de~\eqref{eq:1.1} y \eqref{eq:1.2} junto con la propiedad~\ref{prop:10} se obtiene que $\boxed{d=d^{\prime}}$.
\end{proof}

\end{theorem}

\begin{definition}
Este número $d$ es llamado máximo común divisor de $a$ y $b$ y se denota como $\mcd(a,b)$ o $(a,b)$.
\end{definition}

\begin{remark}
Si el $\mcd(a,b)=1$, entonces $a$ y $b$ son llamados coprimos, primos entre sí (PESI) o primos relativos.
\end{remark}

\subsection{Algunas propiedades del máximo común divisor}

\begin{enumerate}[font={\bfseries},label={\arabic*)}]

\item $(a,b)=(b,a)$.

\item $(a,(b,c))=((a,b),c)$.

\item $(ac,bc)=|c|(a,b)$.

\item $(a,1)=(1,a)=1$.

\item $(a,0)=(0,a)=|a|$.
\end{enumerate}

\begin{theorem}

Si $a\divides bc$ y si $(a,b)=1$, entonces $a\divides c$.
\begin{proof}
Como $(a,b)=1$, entonces existen $\tilde{x},\tilde{y}\in\mathbb{Z}$ de modo que

\begin{equation}\label{eq:1.3}
1=a\tilde{x}+b\tilde{y}
\end{equation}

\noindent
Pero si multiplicamos~\eqref{eq:1.3} por $c$ resulta
\begin{equation}
c=a(c\tilde{x})+b(c\tilde{y})
\end{equation}
Así, $a\divides cx$ y $a\divides cy$ (explicar).

\end{proof}

\end{theorem}

\section{Números primos}

\begin{definition}

El número $n\in\mathbb{N}$ es llamado número primo si sus divisores positivos son 1 y $n$. Cuando $n$ no es primo, será llamado número compuesto. 

\end{definition}

\begin{theorem}
Cada natural $n>$ 1 o es primo o producto de números primos.

\begin{proof}[Prueba:]
Por inducción sobre $n$.

Para $n=2$ \checkmark

Supongamos que se cumple para $n=2,3,\ldots,k-1$.

Demostraremos para $n=k$.

\begin{enumerate}[font={\bfseries},label={*)}]

\item Si $k$ es un número primo.

\item Si $k$ no es un número primo, entonces $k$ tiene por lo menos un divisor $d>1$, por lo que $k=d\cdot c$ con $1<c<k$ y $1<d<k$.

\end{enumerate}

\noindent
Se cumple la hipótesis para $c$ y $d$, entonces $c$ y $d$ son primos o productos de primos.

\begin{align*}
c&=p_1p_2\cdots p_k&(p_i\colon\text{primo}, k\geq1).\\
d&=q_1q_2\cdots q_m&(q_i\colon\text{primo}, m\geq1).
\end{align*}

\noindent
Así, $n=c\cdot d=p_1p_2\cdots p_kq_1q_2\cdots q_m$ (se cumple la inducción).
\end{proof}
\end{theorem}

\begin{theorem}
Existen infinitos números primos.

\begin{proof}[Prueba:]

Supongamos que $\boxed{\mathbb{P}=p_1p_2\cdots p_k}$ es el conjunto  de todos los números primos que existen. Definimos:
\[N=p_1p_2\cdots p_k+1\]
¿Qué tipo de número es $N$, es un primo o uno compuesto?

\noindent
Claro está que $N$ es mayor que $p_i,\forall i=1,\cdots k$.

$N=p_1p_2\cdots p_k+1=q_1q_2\cdots q_t$.

\[\begin{array}{l@{\quad}cr@{}l}
&& q_i & {}\divides p_1p_2\cdots p_k+1 \\
&& q_i & {}\divides p_1p_2\cdots p_k \\ \cline{2-4}
&& q_i & {}\divides 1\quad(\implies\impliedby)
\end{array}\ (-)\]
$\therefore$ Existen infinitos números primos. %(¿numerable?)
\end{proof}

\end{theorem}

\begin{theorem}
Si $p$ es un número primo y $p\notdivides a$, entonces $(p,a)=1$.

\begin{proof}

Sea $d$ el máximo común divisor de $p$ y $a$ (ya que el teorema~\ref{teo:1.1} nos asegura su existencia), $d=(p,a)$, entonces

\[d\divides p\quad\text{y}\quad d\divides a.\]

\end{proof}

\end{theorem}

\begin{theorem}

Sea $p$ un número primo. Si $p\divides ab$, entonces $p\divides a$ o $p\divides b$.

\begin{proof}[Demostración:]
Supongamos que $p\notdivides a$ ($p\divides a$ \checkmark), entonces $(p,a)=1$, en consecuencia, $p\divides ab$.

\[\colorboxed{DarkRed}{a\divides bc\quad\text{y}\quad(a,b)=1\implies a\divides c.}\]

\noindent
$\therefore p\divides b$.
\end{proof}

\end{theorem}

%Generalización: Si $p\divides a_1a_2\cdots a_n\implies p\divides a_1,\ldots, p\divides a_n$. Donde $p$ es un número primo.
\begin{theorem}

Cada entero $n>1$ se representa de forma única como producto de primos no necesariamente distintos, sin importar el orden.

\begin{proof}[Prueba:]

Por inducción en $n$. Cuando $n=2$ (se cumple: 2,3,\ldots,n-1.)
$n=p_1p_2\cdots,p_s=q_1q_2\cdots q_t$ ($s=t$).
$(s,t\geq 1)$

$p_1\divides q_1q_2\cdots q_t\implies p_1\divides q_1\implies p_1=q_1$.
\end{proof}

\end{theorem}

\begin{remark}

Si se desea representar a $n$ como producto de primos distintos (donde cabe la posibilidad en que se repitan algún número primo), podemos escribir:
\[n={p}^{a_1}_{1}{p}^{a_2}_{2}\cdots{p}^{a_k}_{k}=\prod_{i=1}^{k}{p}^{a_i}_{i}\]

\end{remark}

\begin{theorem}

Si $\displaystyle n=\prod_{i=1}^{r}{p}^{a_i}_{i}$, entonces un divisor de $n$ tiene la forma
\[\prod_{i=1}^{r}={p}^{c_i}_{i},\quad0\leq c_i\leq a_i.\]

\end{theorem}

\begin{remark}

Sea la sucesión de números primos
\[p_1=2, p_2=3, p_3=5, \ldots,\]
entonces $\displaystyle n=\prod_{i=1}^{\infty}{p}^{a_i}_{i},\, a_i\geq0$.

\end{remark}

\begin{theorem}

Sean $\displaystyle a=\prod_{i=1}^{\infty}{p}^{a_i}_{i}$ y $\displaystyle b=\prod_{i=1}^{\infty}{p}^{b_i}_{i}$, entonces el máximo común divisor de $a$ y $b$ es

\[(a,b)=\displaystyle\prod_{i=1}^{\infty}{p}^{c_i}_{i}\geq0\text{, donde }c_i=\min\{a_i,b_i\}\leq a_i,b_i.\]

\begin{proof}[Demostración:]

\begin{enumerate}[font={\bfseries},label={*)}]

\item $\displaystyle\prod_{i=1}^{\infty}{p}^{c_i}_{i}\divides a\quad\wedge\quad\prod_{i=1}^{\infty}{p}^{c_i}_{i}\divides b$.

\item $e\divides a\quad\wedge\quad e\divides b, e=\displaystyle\prod_{i=1}^{\infty}{p}^{e_i}_{i}$.

\end{enumerate}

\noindent
Pero, $e_i\leq a_i$ y $e_i\leq b_i$,

\[\implies e_i\leq\min\{a_i,b_i\}=c_i.\]

\end{proof}

\end{theorem}


\end{document}
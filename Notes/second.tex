\chapter{Funciones aritméticas}

\begin{definition}
Una \emph{función aritmética} es cualquiera función $f\colon\mathbb{Z}^{+}\rightarrow\mathbb{C}$.	
\end{definition}

Algunas funciones aritméticas son:

\begin{enumerate}
	\item Función de M\"{o}bius $\mu(n)$
	\[\mu\colon\mathbb{Z}^{+}\longrightarrow\{-1,0,1\}\]
	\[
	\mu(n)= 
	\begin{cases}
	1 & \text{si } n=1,\\
	{(-1)}^{k}  & \text{si } \dc\text{ con } \alpha_1=\alpha_2=\cdots=\alpha_k=1;\\
	0&\text{en otro caso.}
	\end{cases}
	\]
\end{enumerate}

\begin{remark} $\mu(n)=0$ si y solo si $n$ posee un divisor cuadrado perfecto mayor que 1.
\end{remark}
\begin{example}
	Algunos valores de la función $\varphi(n)\colon$
	
\begin{table}[H]
	\centering
	\begin{tabular}{ccccccccccc}
		$n\colon$ &1 &2 &3 & 4 & 5 & 6 & 7 & 8 &9 & 10\\[0.1cm]
		$\mu(n)\colon$ &1 &-1 & -1 & 0 & -1 & 1 & -1 & 0 & 0 & 1
	\end{tabular}
\end{table}

\end{example}

\begin{theorem}
	Si $n\geq1$ y $d>0$, entonces $\displaystyle\sum_{d\divides n}\mu(d)= 
	\begin{cases}
	1 & \text{si } n=1,\\
	0 &\text{si } n>1.
	\end{cases}$

\begin{proof}[Demostración]
	\begin{enumerate}
		\item Si $n=1\colon$ $\underbrace{\mu(1)}_{\displaystyle\sum_{d\divides 1}\mu(d)}=1$ \checkmark
		
		\item Si $n>1$, entonces $\dc$. Si $d\divides n$ y $d$ posee un factor cuadrado perfecto mayor que 1, entonces $\mu(d)=0$.
		\[
		\sum_{d\divides n}\mu(d)=\underbrace{\cancelto{0}{\sum_{d\divides n}\mu(d)}}_{\displaystyle d\text{ posee algún factor }k^{2}>1}+\underbrace{\sum_{d\divides n}\mu(d)}_{\displaystyle d\text{ no posee factor }k^{2}>1}
		\]
		$d\divides n$ y $d$ no poseen factor $k^{2}>1$.
		
		$\displaystyle\implies d=p_{i1}p_{i2}\cdots p_{is}$
		\begin{align*}
		\implies \sum_{d\divides n}\mu(d)
		&=\mu(1)+\mu(p_1)+\mu(p_2)+\cdots+\mu(p_k)+&\\
		& \mu(p_1p_2)+\cdots+\mu(p_{k-1}p_{k})+\cdots\mu(p_1p_2\cdots p_k)&\\
		&=1\binom{k}{0}+(-1)\binom{k}{1}+{(-1)}^{2}\binom{k}{2}+\cdots+{(-1)}^{k}\binom{k}{k}&\\
		&={(1+(-1))}^{k}=0.&
		\end{align*}
	\end{enumerate}
\end{proof}
\end{theorem}

\begin{enumerate}
	\item Función indicador de Euler.
	\[\varphi\colon\mathbb{Z}^{+}\longrightarrow\mathbb{Z}^{+}\]
\end{enumerate}

$\varphi(n)\colon$ cantidad de números menores o iguales que $n$ y coprimos con $n$.

\begin{example}
	
\begin{table}[H]
	\centering
	\begin{tabular}{ccccccccccc}
		$n\colon$ &1 &2 &3 & 4 & 5 & 6 & 7 & 8 &9 & 10\\[0.1cm]
		$\varphi(n)\colon$ &1 &1 & 2 &2 & 4 & 2 & 6 & 4 & 6 & 4
	\end{tabular}
\end{table}	

\end{example}

\begin{theorem}
	Si $n\geq1$, entonces $\displaystyle\sum_{d\divides n}\varphi(d)=n$.
\begin{proof}[Demostración]
	Para cada $d$ tal que $d\divides n$.
	Sea: $\mathcal{A}_d\coloneq\{k\colon\mcd(k,n)=d, 1\leq k\leq n\}$.
	
	$a\in\mathcal{A}_d, \mathcal{A}_{d^{\prime}} (d\neq d^{\prime})$
	
	$\implies \mcd(a,n)=d, \mcd(a,n)=d^{\prime} (\implies\impliedby)$.
	
	$\implies \mathcal{A}_{d}\bigcap\mathcal{A}_{d^{\prime}}=\emptyset (\forall d\neq d^{\prime})$
	
	Sea $f(d)$ la cantidad de elementos de $\mathcal{A}_{d}$.
	%inclusion derecha
	\[\bigcup_{d\divides n}\mathcal{A}_{d}=\{1,2,\ldots,n\}.%={\{i\}}_{i=1}^{n}
	\]
	Porque si $k\leq n\implies \mcd(k,n)=d$ en donde $d\divides n\implies \mathcal{A}_{d}\subset\bigcup_{d\divides n}\mathcal{A}_{d}\implies\displaystyle\sum_{d\divides n}f(d)=n$.
	%Inclusión izquierda
	Pero $\mcd(k,n)=d\implies\mcd\left(\frac{k}{d},\frac{n}{d}\right)=1$.
	
	\[\mathcal{A}_{d}\coloneq\left\{k\colon\left(\frac{k}{d},\frac{n}{d}\right)=1, 1\leq k\leq n\right\}\]
	
	\[\mathcal{B}_{d}\coloneq\left\{q\colon\left(q,\frac{n}{d}\right)=1, 1\leq q\leq \frac{n}{d}\right\}\]
	
	$\implies|\mathcal{A}_{d}|=|\mathcal{B}_{d}|=f(d)=\varphi\left(\frac{n}{d}\right)$
	
	$\displaystyle\implies\sum_{d\divides n}\varphi\left(\frac{n}{d}\right)=n$.
\end{proof}
\end{theorem}

\section{Relación entre $\mu$ y $\varphi$}

\begin{theorem}
	Si $n\geq1$, entonces $\displaystyle\varphi(n)=\sum_{d\divides n}\mu(d)\left(\frac{n}{d}\right)$.
	\begin{proof}[Demostración]
		\[\varphi(n)=\sum_{k=1}^{n}i(k), \text{ donde } i(k)=\begin{cases}
		1 & \text{si } \mcd(n,k)=1,\\
		0 & \text{si } \mcd(n,k)\neq1.
		\end{cases}\]
		Por el teorema:
		\[\sum_{d\divides\mcd(n,k)}\mu(d)=\begin{cases}
		1 & \text{si }\mcd(n,k)=1,\\
		0 & \text{si } \mcd(n,k)>1.
		\end{cases}\]
		
		\[\sum_{d\divides\mcd(n,k)}\mu(d)=\begin{cases}
		1 & \text{si } i(k)=1,\\
		0 & \text{si } i(k)=0.
		\end{cases}\]
		
		\[\sum_{d\divides\mcd(n,k)}\mu(d)=i(k)\]
		
		\[\implies\varphi(n)=\sum_{d\divides\mcd(n,k)}\mu(d)=\sum_{k=1}^{n}\sum_{\substack{d\divides n\\
		d\divides k}}\mu(d)\]
	
		Fijamos un divisor $d$ en $n$. Entonces, el divisor $d$ de $n$ aparecerá siempre y cuando $k$ sea múltiplo de $d$ $(k=qd)$. Por lo tanto,
		\[d\leq k\leq n\implies 1\leq q\leq\frac{n}{d}.\]
		Hay $\dfrac{n}{d}$ múltiplos de $k$.
		\[\implies\varphi(n)=\sum_{d\divides n}\mu(d)\frac{n}{d}.\]
	\end{proof}
\end{theorem}

\subsection{Fórmula para $\varphi(n)$}

\begin{theorem}
	Si $n>1$, entonces $\displaystyle\varphi(n)=n\prod\left(1-\frac{1}{p}\right)$.
	\begin{proof}[Demostración]
		Usar el principio de inclusión y exclusión. Sean $\mathcal{A}_1,\mathcal{A}_2,\mathcal{A}_3,\ldots,\mathcal{A}_k$ conjuntos (puede haber algún conjunto vacío).
		
		\[
		\displaystyle\implies\left|\bigcup^{k}_{i=1}\mathcal{A}_{i}\right|=\sum_{i=1}^{k}|\mathcal{A}_{i}|
		\]
	\end{proof}
\end{theorem}